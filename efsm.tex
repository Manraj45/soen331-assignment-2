\documentclass[12pt]{article}
\usepackage[top=1in,bottom=1in,left=1in,right=1in]{geometry}
\usepackage{alltt}
\usepackage{array}	
\usepackage{graphicx}
\usepackage{tabularx}
\usepackage{verbatim}
\usepackage{setspace}
\usepackage{listings}
\usepackage{amssymb,amsmath, amsthm}

\title{SOEN331: Introduction to Formal Methods\\for Software Engineering\\
Assignment 2 on Extended Finite State Machines}
\author{Samuel Huang, Manraj Rai, Simon Lim, Lauren Lim}
\date{March 6, 2020}

\begin{document}
\begin{spacing}{1.5}

\maketitle
\newpage
\section{Driverless Car formal specification}
\noindent The EFSM of the driverless car is the tuple $S = (Q, \Sigma_1, \Sigma_2, q_0, V, \Lambda)$, where\\

\noindent $Q = \{parked, manual,cruise,panic,system idle\}$\\
\noindent $\Sigma_1 = \{manual ~signal, cruise~ signal, park~signal, unforeseen~event, shut~off~ engine, 
\newline start ~system,  check~ destination\}$\\
\noindent $\Sigma_2 = \{system ~activated/engine~ idle, beep,  hazard~ signal~turned ~on, hazard~ off
\}$\\
\noindent $q_0: parked$\\
\noindent $V: isDestinationSet= \{true,false\}, isEngineIdle= \{true,false\}, isCarStopped = \{true,false\},isDriving = \{true, false\}, hasArrivedToDestination = \{true, false\}$\\
\noindent $\Lambda$: Transition specifications\\

\indent 1. $\rightarrow system~idle$\\
\indent 2. $system~idle \xrightarrow {\text { start system [isEngineIdle is true]}} parked$\\
\indent 3. $ parked\xrightarrow {\text { manual ~signal }} manual$\\
\indent 4. $ parked\xrightarrow {\text { cruise ~signal [isDestinationSet is true] }} cruise$\\
\indent 5. $ parked\xrightarrow {\text { cruise ~signal [isDestinationSet is false] }} parked$\\
\indent 6. $ manual\xrightarrow {\text { cruise ~signal [isDriving is true;isDestinationSet is true] }} manual$\\
\indent 7. $ cruise\xrightarrow {\text { manual ~signal [isDriving true;isDestinationSet is true] }} cruise$\\
\indent 8. $ manual\xrightarrow {\text { park ~signal [isCarStopped is true] }} parked$\\
\indent 9. $ parked\xrightarrow {\text { unforeseen ~event/hazard~signal~turned~on}} panic$\\
\indent 10. $ cruise\xrightarrow {\text {check~destination[hasArrivedToDestination is true]/engine ~idle}} parked$\\
\indent 11. $ panic \xrightarrow {\text { parked~ signal/hazard~off}} parked$\\
\indent 12. $ cruise\xrightarrow {\text { manual~ signal}} manual$\\
\indent 13. $ parked\xrightarrow {\text { shut~off~engine}} system~idle$\\


\noindent The UML state diagram is shown in Figure 1.

\newpage

\subsection{Manual mode}

\noindent The EFSM of the manual mode is the tuple $S = (Q, \Sigma_1, \Sigma_2, q_0, V, \Lambda)$, where\\

\noindent $Q = \{manual, brake\}$\\
\noindent $\Sigma_1 = \{accelerate~signal, decelerate~signal, brake~signal\}$\\
\noindent $\Sigma_2 = \{accelerate, decelerate\}$\\
\noindent $q_0: manual$\\
\noindent $V: speed: \mathbb{R+} $\\
\noindent $\Lambda$: Transition specifications\\
\indent 1. $\rightarrow manual$\\
\indent 2. $manual \xrightarrow {\text { accelerate signal / accelerate}} manual$\\
\indent 3. $manual \xrightarrow {\text { decelerate signal / decelerate}} manual$\\
\indent 4. $manual \xrightarrow {\text { brake signal / speed = 0}} brake$\\
\indent 5. $brake \xrightarrow {\text { accelerate signal / accelerate}} manual$\\


\noindent The UML state diagram is shown in Figure 2.

\newpage
\subsection{Cruise}
\subsubsection{Maintaining desired speed}

\noindent The EFSM of the cruise (maintaining desired speed) ( is the tuple $S = (Q, \Sigma_1, \Sigma_2, q_0, V, \Lambda)$, where\\

\noindent $Q = \{accelerating,~decelerating,~optimal~speed,~initial speed,~optimal~speed\}$\\
\noindent $\Sigma_1 = \{accelerate~signal,~decelerate~signal,~ verify~current\}$\\
\noindent $\Sigma_2 = \{accelerate,~decelerate,~maintain~current~speed\}$\\
\noindent $q_0: initial~speed$\\
\noindent $V: default~speed: \mathbb R+, current~speed: \mathbb R+$\\
\noindent $\Lambda$: Transition specifications\\
\indent 1. $\rightarrow initial~speed$\\
\indent 2. $initial~speed \xrightarrow {\text { accelerate signal [current speed \textless~default speed - 5\% of default speed]}~/ ~accelerate} accelerating$\\
\indent 3. $initial~speed \xrightarrow {\text { decelerate signal [current speed \textgreater~default speed + 5\% of default speed]}~/ ~decelerate} decelerating$\\
\indent 4. $initial~speed \xrightarrow {\text { verify current speed [95\% of default speed \textless~current speed \textless~105\% of default speed}~/~maintain~current~speed} optimal speed$\\
\indent 5. $accelerate \xrightarrow {\text { verify current speed [95\% of default speed \textless~105\% of default speed}~/~maintain~current~speed} optimal speed$\\
\indent 6. $decelerate \xrightarrow {\text { verify current speed [95\% of default speed \textless~105\% of default speed}~/~maintain~current~speed} optimal~speed$\\


\noindent The UML state diagram is shown in Figure 3.

\newpage

\subsubsection{Avoiding obstacles}

\noindent The EFSM of the cruise (avoiding obstacles) ( is the tuple $S = (Q, \Sigma_1, \Sigma_2, q_0, V, \Lambda)$, where\\

\noindent $Q = \{cruising,~changing~lane,~tailing\}$\\
\noindent $\Sigma_1 = \{verify~distance~from~obstacle,~change~lane~signal, ~verify~changing~lane\}$\\
\noindent $\Sigma_2 = \{maintain~speed,~reduce~speed,~change~lane~ left\}$\\
\noindent $q_0: cruising$\\
\noindent $V: threshold~limit: \mathbb R+,~distanceFromObstacle:~ \mathbb R+,~isObstacledDetected=~\{true, false\},\newline~isObstacleMoving=~\{true,false\},~isSafeDistanceMaintained=~\{true,false\},\newline ~isChangeLaneDone=~\{true,false\}$\\
\noindent $\Lambda$: Transition specifications\\
\indent 1. $\rightarrow cruising$\\
\indent 2. $cruising \xrightarrow {\text { verify distance from obstacle [distanceFromObstacle \textgreater~threshold limit]}~/ ~maintain~ speed} cruising$\\
\indent 3. $cruising \xrightarrow {\text {verify distance from obstacle [distanceFromObstacle \textless~threshold limit]}~/ ~reduce~ speed} tailing$\\
\indent 4. $tailing \xrightarrow {\text { verify distance from obstacle [distanceFromObstacle \textgreater~threshold limit]}~/ ~maintain~ speed} cruising$\\
\indent 5. $cruising \xrightarrow {\text { change lane signal [isSafeDistanceMaintained = false or isObstacleMoving = false]}~/ ~change~lane~left} changing lane$\\
\indent 6. $changingLane \xrightarrow {\text {verify changing lane [isChangeLaneDone = true]}} cruising$\\


\noindent The UML state diagram is shown in Figure 4.

\newpage

\subsubsection{Navigation mode}

\noindent The EFSM of the navigation mode is the tuple $S = (Q, \Sigma_1, \Sigma_2, q_0, V, \Lambda)$, where\\

\noindent $Q = \{cruising, changing lane, turning, destination~ahead, destination~arrived\}$\\
\noindent $\Sigma_1 = \{turn~left~signal, turn~right~signal, verify~intersection, verify~turn~done\}$\\
\noindent $\Sigma_2 = \{change~to~left-most~lane, change~to~right-most~lane, turn~left, turn~right, go~straight, signal~destination~ahead, stop\}$\\
\noindent $q_0: cruising$\\
\noindent $V: isIntersection = \{true, false\}, isCloseToDestination = \{true, false\}, isDestination = \{true, false\}, isTurnDone = \{true, false\}, isRightMostLane = \{true, false\}$\\
\noindent $\Lambda$: Transition specifications\\
\indent 1. $\rightarrow cruising$\\
\indent 2. $cruising \xrightarrow {\text { turn~left~signal / change~to~left-most~lane}} changing~lane$\\
\indent 3. $cruising \xrightarrow {\text { turn~right~signal / change~to~right-most~lane}} changing~lane$\\
\indent 4. $cruising \xrightarrow {\text { verify~intersection [isIntersection~is~true; turn~left~signal] / turn~left}} turning$\\
\indent 5. $cruising \xrightarrow {\text { verify~intersection [isIntersection~is~true; turn~right~signal] / turn~right}} turning$\\
\indent 6. $cruising \xrightarrow {\text { verify~intersection [isCloseToDestination~is~true; isRightMostLane~is~true] / signal~destination~ahead}} destination~ahead$\\
\indent 7. $changing~lane \xrightarrow {\text { verify~intersection [isIntersection~is~true; turn~left~signal] / turn~left}} turning$\\
\indent 8. $changing~lane \xrightarrow {\text { verify~intersection [isIntersection~is~false] / go~straight}} cruising$\\
\indent 9. $changing~lane \xrightarrow {\text { verify~intersection [isIntersection~is~true; turn~right~signal] / turn right}} turning$\\
\indent 10. $destination~ahead \xrightarrow {\text { verify~intersection [isDestination~is~true] / stop}} destination~arrived$\\
\indent 11. $turning \xrightarrow {\text { verify~turn~done [isTurnDone~is~true] / go straight}} cruising$\\

\noindent The UML state diagram is shown in Figure 5.

\newpage
\subsubsection{Changing Lane}

\noindent $Q = \{Cruising, ChangingLane, Panic\}$\\
\noindent $\Sigma_1 = \{MoveLeftSignal, MoveRightSignal, MoveLeft, MoveRight, DangerDetection\}$\\
\noindent $\Sigma_2 = \{ChangeLane, StayInLane, PanicOn\}$\\
\noindent $q_0: Cruising$\\
\noindent $V: Lane = \{Target, NotTarget\}, $ LeftLaneAvail  = \{Open, NotOpen\}, $ RightLaneAvail  = \{Open, NotOpen\}, $ LaneLocation  = \{Left, Right\}, $ ObstacleAhead  = \{True, False\}$\\
\noindent $\Lambda$: Transition specifications\\

\indent 1. $\rightarrow Cruising$\\
\indent 2. $Cruising \xrightarrow {\text {MoveLeftSignal [LaneLocation is Left]}} ChangeLane$\\
\indent 3. $ChangeLane \xrightarrow {\text {MoveLeft [Lane is NotTarget; LeftLaneAvail is Open] / ChangeLane}} ChangeLane$\\
\indent 4. $ChangeLane \xrightarrow {\text {MoveLeft [Lane is Target; LeftLaneAvail is Open] / ChangeLane}} Cruising$\\
\indent 5. $ChangeLane \xrightarrow {\text {MoveLeft [LeftLaneAvail  is NotOpen] / StayInLane}} ChangeLane$\\
\indent 6. $Cruising \xrightarrow {\text { MoveRightSignal [LaneLocation is Right]}} ChangeLane$\\
\indent 7. $ChangeLane \xrightarrow {\text {MoveRight [Lane is NotTarget; RightLaneAvail is Open] / ChangeLane}} ChangeLane$\\
\indent 8. $ChangeLane \xrightarrow {\text {MoveRight [Lane is Target; RightLaneAvail is Open] / ChangeLane}} Cruising$\\
\indent 9. $ChangeLane \xrightarrow {\text {MoveRight [RightLaneAvail is NotOpen] / StayInLane}} ChangeLane$\\
\indent 10. $Cruising \xrightarrow {\text {DangerDetection [LeftLaneAvail is NotOpen; RightLaneAvail is NotOpen; ObstacleAhead is True] / (PanicModeOn)}} Panic$\\

\noindent The UML state diagram is shown in Figure 6.

\newpage

\subsection{Panic}

\noindent $Q = \{Default, Panic, Parked\}$\\
\noindent $\Sigma_1 = \{Panic Signal, UnknownIssueSignal\}$\\
\noindent $\Sigma_2 = \{hazard signal turned on, break, hazard signal turned off\}$\\
\noindent $q_0: Default$\\
\noindent $V: UnsolvableIssue = \{True, False\}, $ PanicState = \{True, False\}\\
\noindent $\Lambda$: Transition specifications\\

\indent 1. $\rightarrow Default$\\
\indent 2. $Default \xrightarrow {\text {UnknownIssueSignal [UnsolvableIssue is True] / (break;~hazard~signal~turned on)}} Panic$\\
\indent 3. $Default \xrightarrow {\text {Panic Signal}} Panic$\\
\indent 4. $Panic \xrightarrow {\text {Panic Signal [PanicState is True] /(hazard~signal~turned~off)}} Parked$\\


\noindent The UML state diagram is shown in Figure 7.

\newpage

\section{UML Diagrams}

\begin{figure}[h!]
	\centering
		
		  \caption{Metro.}
  \label{fig:metro-fig}
\end{figure}

\end{spacing}


\end{document}
