\documentclass[12pt]{article}
\usepackage[top=1in,bottom=1in,left=1in,right=1in]{geometry}
\usepackage{alltt}
\usepackage{array}	
\usepackage{graphicx}
\usepackage{tabularx}
\usepackage{verbatim}
\usepackage{setspace}
\usepackage{listings}
\usepackage{amssymb,amsmath, amsthm}

\title{SOEN331: Introduction to Formal Methods\\for Software Engineering\\
Assignment 2 on Extended Finite State Machines}
\author{Samuel Huang, Manraj Rai, Simon Lim, Lauren Lim}
\date{March 6, 2020}

\begin{document}
\begin{spacing}{1.5}

\maketitle

\section{Metro passageway formal specification}

\noindent The EFSM of the metro passageway is the tuple $S = (Q, \Sigma_1, \Sigma_2, q_0, V, \Lambda)$, where\\

\noindent $Q = \{locked, unlocked\}$\\
\noindent $\Sigma_1 = \{request~entry, pass\}$\\
\noindent $\Sigma_2 = \{lock, unlock, beep\}$\\
\noindent $q_0: locked$\\
\noindent $V: ticket = \{valid, invalid\}$\\
\noindent $\Lambda$: Transition specifications\\
\indent 1. $\rightarrow locked$\\
\indent 2. $locked \xrightarrow {\text { request entry [ticket is valid] / (unlock ; beep)}} unlocked$\\
\indent 3. $unlocked \xrightarrow {\text { pass / lock}} locked$\\

\noindent The UML state diagram is shown in Figure~\ref{fig:metro-fig}.

\newpage

\section{Cruise}
\subsection{Cruise - Maintaining desired speed}

\noindent The EFSM of the cruise (maintaining desired speed) ( is the tuple $S = (Q, \Sigma_1, \Sigma_2, q_0, V, \Lambda)$, where\\

\noindent $Q = \{accelerating,~decelerating,~optimal~speed,~initial speed,~optimal~speed\}$\\
\noindent $\Sigma_1 = \{accelerate~signal,~decelerate~signal,~ verify~current\}$\\
\noindent $\Sigma_2 = \{accelerate,~decelerate,~maintain~current~speed\}$\\
\noindent $q_0: initial~speed$\\
\noindent $V: default~speed: \mathbb R, current~speed: \mathbb R$\\
\noindent $\Lambda$: Transition specifications\\
\indent 1. $\rightarrow initial~speed$\\
\indent 2. $initial~speed \xrightarrow {\text { accelerate signal [current speed < default speed - 5\% of default speed]}~/ ~accelerate} accelerating$\\
\indent 3. $initial~speed \xrightarrow {\text { decelerate signal [current speed > default speed + 5\% of default speed]}~/ ~decelerate} decelerating$\\
\indent 4. $initial~speed \xrightarrow {\text { verify current speed [default speed - 5\% of default speed < current speed < default speed + 5\% of default speed}~/~maintain~current~speed} optimal speed$\\
\indent 5. $accelerate \xrightarrow {\text { verify current speed [default speed - 5\% of default speed < default speed + 5\% of default speed}~/~maintain~current~speed} optimal speed$\\
\indent 6. $decelerate \xrightarrow {\text { verify current speed [default speed - 5\% of default speed < default speed + 5\% of default speed}~/~maintain~current~speed} optimal speed$\\


\noindent The UML state diagram is shown in Figure~\ref{fig:metro-fig}.
\newpage

\subsection{Cruise - Avoiding obstacles}

\noindent The EFSM of the cruise (avoiding obstacles) ( is the tuple $S = (Q, \Sigma_1, \Sigma_2, q_0, V, \Lambda)$, where\\

\noindent $Q = \{cruising,~changing~lane,~tailing\}$\\
\noindent $\Sigma_1 = \{verify~distance~from~obstacle\}$\\
\noindent $\Sigma_2 = \{maintain~speed,~reduce~speed,~change~one lane~to~the~left\}$\\
\noindent $q_0: cruising$\\
\noindent $V: threshold~limit: \mathbb R,~distanceFromObstacle:~ \mathbb R,~isObstacledDetected=~\{true, false\},\newline~isObstacleMoving=~\{true,false\},~isSafeDistanceMaintained=~\{true,false\},~isChangingLaneDone=~\{true,false\}$\\
\noindent $\Lambda$: Transition specifications\\
\indent 1. $\rightarrow cruising$\\
\indent 2. $cruising \xrightarrow {\text { verify distance from obstacle [distanceFromObstacle > threshold limit}~/ ~maintain speed} cruising$\\
\indent 3. $cruising \xrightarrow {\text { verify distance from obstacle [distanceFromObstacle < threshold limit}~/ ~reduce speed} tailing$\\
\indent 4. $tailing \xrightarrow {\text { verify distance from obstacle [distanceFromObstacle > threshold limit}~/ ~maintain speed} cruising$\\
\indent 5. $cruising \xrightarrow {\text { change lane signal [isSafeDistanceMaintained = false or isObstacleMoving = false}~/ ~change one lane to the left} changing lane$\\
\indent 6. $changingLane \xrightarrow {\text { change lane signal [isSafeDistanceMaintained = false or isObstacleMoving = false}~/ ~change one lane to the left} changing lane$\\


\noindent The UML state diagram is shown in Figure~\ref{fig:metro-fig}.

\newpage
\section{UML state diagrams}

\begin{figure}[h!]
	\centering
		\includegraphics[width=0.8\textwidth]{./figures/eps/metro.eps}
		  \caption{Metro.}
  \label{fig:metro-fig}
\end{figure}

\end{spacing}


\end{document}
